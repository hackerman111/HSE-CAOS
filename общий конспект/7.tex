\clearpage

\section{Адресация памяти в x86-64}
Продолжаем изучение \gls{asm}. Ключевой темой является работа с памятью. В прошлый раз мы установили, что для обращения к памяти (разыменования) используется синтаксис с квадратными скобками.

\subsection{Синтаксис Scale-Index-Base (SIB)}
Общий синтаксис адресации памяти в 64-битном режиме (в \gls{intel-syntax}) выглядит следующим образом:
$$ [rbase + rindex \times scale + displacement] $$
где:
\begin{itemize}
    \item \texttt{rbase} — базовый регистр.
    \item \texttt{rindex} — регистр-индекс.
    \item \texttt{scale} — множитель (масштаб) для индекса. Допустимые значения: $scale \in \{1, 2, 4, 8\}$.
    \item \texttt{displacement} — константное смещение (сдвиг).
\end{itemize}

Этот синтаксис был разработан для удобной работы с массивами и структурами. Например, \texttt{rbase} может хранить адрес начала массива, \texttt{rindex} — индекс элемента, \texttt{scale} — размер одного элемента (e.g., 8 байт для \texttt{uint64\_t}), а \texttt{displacement} — сдвиг до нужного поля внутри структуры.

\begin{lstlisting}[language={[x86masm]Assembler}, caption={Примеры SIB-адресации}, label={lst:sib_examples}]
; * (uint64_t*)(rax + 8 * rdx) = rcx
; (rax = base, rdx = index, 8 = scale)
mov [rax + rdx * 8], rcx

; * (uint64_t*)(rbx + rbp + 32) = rax
; (rbx = base, rbp = index, 1 = scale (default), 32 = displacement)
mov [rbx + rbp + 32], rax
\end{lstlisting}

\subsection{Указание размера операнда}
В \lstref{lst:sib_examples} ассемблер мог угадать размер операции (64 бита) по размеру регистра \texttt{rcx} или \texttt{rax}. Однако при работе с константами возникает неоднозначность.

\begin{lstlisting}[language={[x86masm]Assembler}, caption={Неоднозначность размера}, label={lst:ambiguity}]
mov [rax], 0 ; OSHIBKA: Neizvesten razmer: 1, 2, 4 ili 8 bayt?
\end{lstlisting}

Компилятор ассемблера не знает, какой размер данных вы намереваетесь записать. Для явного указания размера используются специальные директивы:
\begin{itemize}
    \item \texttt{BYTE PTR} — 8 бит (1 байт).
    \item \texttt{WORD PTR} — 16 бит (2 байта).
    \item \texttt{DWORD PTR} — 32 бита (4 байта).
    \item \texttt{QWORD PTR} — 64 бита (8 байт).
\end{itemize}

\begin{lstlisting}[language={[x86masm]Assembler}, caption={Явное указание размера (32 бита)}, label={lst:dword_ptr}]
; * (uint32_t*)rax = 0
mov DWORD PTR [rax], 0
\end{lstlisting}

\begin{summarybox}
\begin{itemize}
    \item Адресация SIB ($[base + index \times scale + disp]$) — основной механизм доступа к памяти.
    \item $scale$ ограничен значениями $\{1, 2, 4, 8\}$.
    \item При неоднозначности (например, при записи константы) размер операции нужно указывать явно (e.g., \texttt{DWORD PTR}).
\end{itemize}
\end{summarybox}

\section{Инструкция LEA (Load Effective Address)}
Инструкция \gls{lea} — один из самых полезных и часто используемых инструментов в \gls{asm}.

\begin{definitionbox}{LEA (Load Effective Address)}
Инструкция \texttt{lea} \textbf{вычисляет} адрес, используя синтаксис SIB, но \textbf{не разыменовывает} его. Вместо этого она записывает вычисленный адрес в регистр-приемник.
\end{definitionbox}

\begin{lstlisting}[language={[x86masm]Assembler}, caption={Сравнение MOV и LEA}, label={lst:mov_vs_lea}]
; MOV: Prochitat' 8 bayt po adresu [rax] i polozhit' v rdx
; rdx = * (uint64_t*)rax
mov rdx, [rax]

; LEA: Vychislit' adres (v etom sluchae prosto rax) i polozhit' v rdx
; rdx = rax
lea rdx, [rax]
\end{lstlisting}

Основное применение \gls{lea} — это вычисление адресов, но благодаря своей способности выполнять сложение и умножение (на 1, 2, 4, 8), она стала мощным инструментом для арифметических вычислений.

\begin{lstlisting}[language={[x86masm]Assembler}, caption={LEA для вычисления адреса}, label={lst:lea_arithmetic}]
; rdx = rax + 4 * rbx + 16
lea rdx, [rax + rbx * 4 + 0x10]
\end{lstlisting}

\subsection{LEA как оптимизация компилятора}
Компиляторы часто используют \gls{lea} для выполнения простых арифметических операций, так как \gls{lea} часто выполняется быстрее, чем инструкции умножения (такие как \texttt{imul}).
Например, для компиляции функции \texttt{a * 3}:
\begin{lstlisting}[language=C, caption={C++ код для умножения на 3}, label={lst:mul3_cpp}]
uint64_t Mul3(uint64_t a) {
    return a * 3;
}
\end{lstlisting}

Компилятор (\texttt{g++ -O2}) сгенерирует следующий код (\lstref{lst:mul3_asm}), используя \gls{lea} вместо умножения. В Linux (System V AMD64 ABI) первый аргумент (\texttt{a}) передается в регистре \texttt{rdi}, а возвращаемое значение — в \texttt{rax}.
$$ a \times 3 = a \times (1 + 2) = a + a \times 2 $$
Этот паттерн идеально ложится в SIB-адресацию: $[rdi + rdi \times 2]$.

\begin{lstlisting}[language={[x86masm]Assembler}, caption={Результат компиляции Mul3 (objdump)}, label={lst:mul3_asm}]
0000000000000000 <Mul3(unsigned long)>:
   0: f3 0f 1e fa             endbr64            ; Zashchitnaya instruktsiya
   4: 48 8d 04 7f             lea    rax,[rdi+rdi*2] ; rax = rdi + rdi * 2
   8: c3                      ret
\end{lstlisting}

\begin{summarybox}
\begin{itemize}
    \item \texttt{lea} вычисляет адрес, но не читает память.
    \item Это мощный инструмент для компактных арифметических вычислений, часто используемый компиляторами.
\end{itemize}
\end{summarybox}

\section{Работа со стеком и локальными переменными}
У процессора ограниченное количество регистров. При вызове функции (инструкция \texttt{call}) возникает проблема: как сохранить значения локальных переменных, если вызываемая функция может перезаписать ("испортить") регистры?

\subsection{Проблема: Callee-clobbered регистры}
Согласно соглашениям о вызовах (Calling Conventions), большинство регистров (как \texttt{rax}, \texttt{rdx}, \texttt{rdi} и т.д.) являются \textit{callee-clobbered} — вызываемая функция (\textit{callee}) имеет право изменять их без восстановления.

Рассмотрим код, где мы храним \texttt{a} и \texttt{b} в регистрах:
\begin{lstlisting}[language={[x86masm]Assembler}, caption={Проблема сохранения локальных переменных}, label={lst:local_vars_problem}]
mov rax, 1   ; a = 1
mov rdx, 2   ; b = 2
call f       ; f()
add rax, rdx ; ??? (rdx mozhet byt' isporchen funktsiey f)
ret
\end{lstlisting}
После возврата из \texttt{f}, мы не можем полагаться на то, что в \texttt{rdx} все еще лежит 2.

\subsection{Решение 1: Сохранение на стеке}
Основной механизм для сохранения локальных переменных — это \gls{stack-frame}. Мы можем "зарезервировать" место на стеке, сдвинув указатель стека \gls{rsp}, и сохранить туда наши значения.

\begin{lstlisting}[language={[x86masm]Assembler}, caption={Использование стека для локальных переменных}, label={lst:local_vars_stack}]
mov rax, 1           ; a = 1
mov rdx, 2           ; b = 2

sub rsp, 16          ; Rezerviruem 16 bayt na steke
mov [rsp + 8], rdx   ; Sokhranyaem b (po smeshcheniyu 8)
mov [rsp], rax       ; Sokhranyaem a (na vershinu steka)

call f               ; f()

; VOSSTANOVLENIE
mov rax, [rsp]       ; Vosstanavlivaem a
mov rdx, [rsp + 8]   ; Vosstanavlivaem b
add rsp, 16          ; Osvobozhdaem mesto na steke

add rax, rdx         ; Teper' bezopasno
ret
\end{lstlisting}

\subsection{Решение 2: Callee-saved регистры}
Некоторые регистры, напротив, являются \textit{callee-saved} (например, \texttt{rbx}, \texttt{rbp}). Это означает, что если вызываемая функция хочет их использовать, она \textit{обязана} сохранить их значение (обычно на стеке) и восстановить перед выходом (\texttt{ret}).
Компиляторы используют это для оптимизации.

Рассмотрим C++ код:
\begin{lstlisting}[language=C, caption={C++ код Sum()}, label={lst:sum_cpp}]
uint64_t f();
uint64_t g();
uint64_t Sum() {
    return f() + g();
}
\end{lstlisting}
Чтобы вычислить \texttt{g()}, нужно сначала вызвать \texttt{f()}, но результат \texttt{f()} (который вернется в \texttt{rax}) будет перезаписан результатом \texttt{g()}.
Компилятор (\lstref{lst:sum_asm}) решает эту проблему, сохраняя результат \texttt{f()} в callee-saved регистре \texttt{rbx}.

\begin{lstlisting}[language={[x86masm]Assembler}, caption={Дизассемблированный код Sum() (g++ -O2)}, label={lst:sum_asm}]
<Sum()>:
   push   rbx            ; 1. Sokhranit' staroe znachenie rbx
   call   <f()>          ; 2. Vyzvat' f(). Rezul'tat v rax
   mov    rbx, rax       ; 3. Spryatat' rezul'tat f() v rbx
   call   <g()>          ; 4. Vyzvat' g(). Rezul'tat v rax
   add    rax, rbx       ; 5. rax = rax + rbx (rezul'tat g() + rezul'tat f())
   pop    rbx            ; 6. Vosstanovit' staroe znachenie rbx
   ret
\end{lstlisting}

\begin{notebox}
В \lstref{lst:sum_asm} мы видим \texttt{call} на адрес вроде \texttt{<Sum()+0xa>} (в реальном \texttt{objdump} это часто \texttt{call 0}). Это \gls{relocation}. На этапе компиляции адрес функции \texttt{f} еще неизвестен. Компилятор оставляет "дырку" (часто 0), а компоновщик (linker) на финальном этапе сборки подставляет в это место реальный адрес функции.
\end{notebox}

\section{Фреймовые указатели (Frame Pointers)}
В \lstref{lst:local_vars_stack} мы вручную двигали \gls{rsp} (\texttt{sub rsp, 16}) и обращались к переменным относительно \gls{rsp} (\texttt{[rsp + 8]}). Это работает, но усложняет отладку и трассировку стека.

\begin{definitionbox}{Фреймовый указатель (RBP)}
\gls{rbp} (Base Pointer) — это регистр, который по соглашению используется для хранения адреса \textit{начала} текущего \gls{stack-frame}. Это обеспечивает "стабильный" якорь для доступа к локальным переменным, даже если \gls{rsp} постоянно движется (например, при \texttt{push}/\texttt{pop}).
\end{definitionbox}

Для использования \gls{rbp} применяется стандартный \textbf{пролог} (в начале функции) и \textbf{эпилог} (в конце).

\begin{lstlisting}[language={[x86masm]Assembler}, caption={Стандартный пролог и эпилог функции}, label={lst:rbp_prolog}]
f:
    ; --- PROLOG ---
    push rbp         ; 1. Sokhranit' RBP predydushchey funktsii na stek
    mov  rbp, rsp    ; 2. Zapomnit' tekushchuyu vershinu steka kak bazu (nachalo)
                     ;    nashego freyma. Teper' RBP stabilen.
    
    ; --- Telo funktsii ---
    ; Mesto dlya lokal'nykh peremennykh vydelyaetsya zdes'
    ; sub rsp, 32    ; (vydelit' 32 bayta)
    ; Dostup k peremennym idet otnositel'no RBP:
    ; mov [rbp - 8], rax
    
    ; --- EPILOG ---
    mov  rsp, rbp    ; 1. Osvobodit' vse lokal'nye peremennye, vernuv rsp k baze
    pop  rbp         ; 2. Vosstanovit' RBP predydushchey funktsii
    ret
\end{lstlisting}

\subsection{Структура стека с RBP}
Когда каждая функция использует этот пролог, значения \gls{rbp} на стеке образуют \textbf{односвязный список}. Каждое сохраненное значение \gls{rbp} указывает на \gls{rbp} предыдущей (вызвавшей) функции.

\begin{figure}[h]
  \centering
  \begin{tikzpicture}[node distance=0mm]
    \node[membox] (frame3_locals) {Локальные переменные F3};
    \node[fpbox, right=of frame3_locals] (frame3_fp) {RBP(F2)};
    \node[retbox, right=of frame3_fp] (frame3_ret) {Адрес возврата в F2};
    \node[membox, right=of frame3_ret] (frame2_locals) {...};
    \node[fpbox, right=of frame2_locals] (frame2_fp) {RBP(F1)};
    \node[retbox, right=of frame2_fp] (frame2_ret) {Адрес возврата в F1};
    \node[membox, right=of frame2_ret] (frame1_locals) {...};
    
    % RBP Pointers
    \draw[arrow, thick] (frame3_fp.south) .. controls +(south:1cm) and +(south:1cm) .. (frame2_fp.south);
    
    % RSP/RBP labels
    \node[above=0.3cm of frame3_locals] (rsp_label) {\small \texttt{RSP} (растет влево)};
    \draw[arrow, thick, AccentDark] (rsp_label) -- (frame3_locals.north);
    
    \node[above=0.8cm of frame3_fp] (rbp_label) {\small \texttt{RBP} (текущий)};
    \draw[arrow, thick, AccentDark] (rbp_label) -- (frame3_fp.north);
    
    \node[below=1.6cm of frame3_ret] (list_label) {\small Связный список фреймов};
  \end{tikzpicture}
  \caption{Структура стека при использовании фреймовых указателей (\texttt{RBP})}
  \label{fig:frame_pointers}
\end{figure}

Это позволяет отладчикам и другим инструментам легко "разматывать" стек (stack unwinding) и строить трассировку вызовов (stack trace).

\begin{notebox}
Использование \gls{rbp} как фреймового указателя — это \textit{соглашение}. Оно требует одного лишнего регистра и нескольких инструкций в прологе/эпилоге.
Современные компиляторы (\texttt{g++ -O2}) по умолчанию часто отключают фреймовые указатели (\texttt{-fomit-frame-pointer}) для оптимизации. Вместо этого они генерируют специальную отладочную информацию (DWARF), которая позволяет разматывать стек, зная только \gls{rip}.
\end{notebox}

\section{Секции данных в ассемблере}
Ассемблерный код и данные не хранятся вперемешку. Они организованы в секции, которые сообщают операционной системе, как их следует загружать в память.

\begin{itemize}
    \item \textbf{.text} — Код (инструкции). [cite: 216] Загружается с правами \texttt{Read-Only} и \texttt{Execute} (RX). [cite: 499]
    \item \gls{rodata} — Данные только для чтения. [cite: 217] (e.g., строковые литералы, константы). Загружаются с правами \texttt{Read-Only} (R). [cite: 501]
    \item \gls{data} — Инициализированные данные. [cite: 217] (e.g., глобальные переменные с начальным значением). Загружаются с правами \texttt{Read-Write} (RW). [cite: 505]
    \item \gls{bss} — Неинициализированные данные. [cite: 218] (e.g., \texttt{int x;}). Эти данные \textit{не хранятся} в исполняемом файле, файл хранит только их размер. При загрузке ОС выделяет память и \textit{обнуляет} ее. [cite: 507]
\end{itemize}

\subsection{Директивы ассемблера для данных}
Мы можем явно указать, в какую секцию помещать байты, с помощью директив.

\begin{lstlisting}[language={[x86masm]Assembler}, caption={Пример секции .rodata (строка "Hello!")}, label={lst:rodata_example}]
; Ob"yavlyaem sektsiyu .rodata
.section .rodata

.global s1  ; Delaem metku s1 vidimoy dlya linkera
s1:
    .byte 0x48, 0x65 ; 'H', 'e'
    .ascii "ll"      ; 'l', 'l'
    .asciz "o!"      ; 'o', '!', i nulevoy bayt (terminator)
\end{lstlisting}
В \lstref{lst:rodata_example} метка \texttt{s1} указывает на строку \texttt{"Hello!\textbackslash0"}.

Другие директивы для инициализации данных в секциях \gls{data} или \gls{rodata}:
\begin{lstlisting}[language={[x86masm]Assembler}, caption={Директивы .data и .bss}, label={lst:data_bss}]
.data
    val1: .byte 0x10                   ; 1 bayt
    val2: .short 0x1234                ; 2 bayta
    val3: .long 0x12345678             ; 4 bayta
    val4: .quad 0x1122334455667788     ; 8 bayt
    
    ; Zapolnit' 32 bayta znacheniem 0xFF
    buffer: .space 32, 0xFF

.bss
    ; Zarezervirovat' 1024 bayta (budut obnuleny)
    big_buffer: .skip 1024
\end{lstlisting}

\begin{notebox}
\textbf{Все есть байты.} В конечном счете, ассемблер просто транслирует мнемоники инструкций в байты в секции \texttt{.text}. Можно написать функцию, используя только директиву \texttt{.byte}, если знать машинные коды.
\end{notebox}

\section{Флаги процессора и условные переходы}
Большинство арифметических и логических инструкций (ALU) изменяют специальный регистр флагов (EFLAGS/RFLAGS). Условные переходы (\texttt{jcc}) анализируют эти флаги.

Основные флаги, интересующие нас:
\begin{itemize}
    \item \gls{cf} (Carry Flag) — Установлен, если произошел перенос/заём из старшего бита (индикатор \textbf{беззнакового} переполнения). [cite: 313, 545]
    \item \gls{zf} (Zero Flag) — Установлен, если результат операции равен нулю. [cite: 314, 546]
    \item \gls{sf} (Sign Flag) — Установлен, если старший бит результата равен 1 (индикатор \textbf{отрицательного} числа в знаковой интерпретации). [cite: 315, 546]
    \item \gls{of} (Overflow Flag) — Установлен, если произошло \textbf{знаковое} переполнение (e.g., $100 + 100$ дало отрицательный результат в 8-битном знаковом представлении). [cite: 316, 547]
\end{itemize}

\begin{lstlisting}[language=bash, caption={Примеры установки флагов (4-битная арифметика)}, label={lst:flags_example}]
# 0b0001 + 0b0111 = 0b1000 (1 + 7 = 8)
# Rezul'tat (8) imeet bit znaka (SF=1).
# Proizoshlo znakovoe perepolnenie (1+7 != -8) (OF=1).
# Flagi: SF, OF

# 0b1001 + 0b0111 = 0b0000 (s perenosom) (9 + 7 = 16)
# Rezul'tat 0 (ZF=1).
# Proizoshlo bezznakovoe perepolnenie (CF=1).
# Flagi: ZF, CF

# 0b0010 - 0b0011 = 0b1111 (s zaemom) (2 - 3 = -1)
# Rezul'tat -1 (SF=1).
# Proizoshel bezznakovyy zaem (CF=1).
# Flagi: CF, SF
\end{lstlisting}

Инструкция \texttt{cmp} (compare) — это, по сути, \texttt{sub}, которая не сохраняет результат, а только устанавливает флаги.

После \texttt{cmp} (или \texttt{add}, \texttt{sub}, \texttt{and}...) используются инструкции условного перехода \texttt{jcc}:
\begin{itemize}
    \item \texttt{je} / \texttt{jz} — Jump if Equal / Jump if Zero (проверяет \gls{zf}=1).
    \item \texttt{jne} / \texttt{jnz} — Jump if Not Equal / Jump if Not Zero (\gls{zf}=0).
    \item \texttt{js} — Jump if Sign (\gls{sf}=1).
    \item \texttt{ja} — Jump if Above (беззнаковое "больше") (проверяет \gls{cf}=0 и \gls{zf}=0).
    \item \texttt{jg} — Jump if Greater (знаковое "больше") (проверяет \gls{sf}=\gls{of} и \gls{zf}=0).
\end{itemize}

\section{Взаимодействие ассемблера и C/C++}
Можно смешивать код на C/C++ и \gls{asm} в одной программе, если соблюдать соглашения.

\subsection{Позиционно-независимый код (PIC) и RIP-адресация}
При попытке получить доступ к глобальной переменной (e.g., из C++ или секции \gls{data}) возникает проблема:
\begin{lstlisting}[language={[x86masm]Assembler}, caption={Наивный доступ к глобальной переменной}, label={lst:bad_global}]
.data
my_var: .quad 123

.text
; OSHIBKA: Ne budet rabotat' v sovremennykh OS
mov rax, [my_var]
\end{lstlisting}
Проблема в том, что в современных ОС из соображений безопасности (ASLR — Address Space Layout Randomization) программа загружается в память по \textit{случайному} адресу. [cite: 587-589] Мы не знаем абсолютный адрес \texttt{my\_var} на этапе компиляции. [cite: 583]

Решение — \gls{pic} (\gls{pic}). Код не должен полагаться на абсолютные адреса, а только на \textit{относительные}.

\begin{definitionbox}{RIP-относительная адресация}
В 64-битном режиме можно адресовать данные \textit{относительно указателя инструкции} (\gls{rip}). Так как \gls{rip} всегда указывает на следующую исполняемую инструкцию, а \texttt{my\_var} находится на \textit{неизменном} расстоянии от этой инструкции (весь код и данные сдвигаются вместе), этот сдвиг остается константой. [cite: 593-594]
\end{definitionbox}

\begin{lstlisting}[language={[x86masm]Assembler}, caption={Корректный доступ к глобальной переменной (PIC)}, label={lst:pic_global}]
extern c ; Ob"yavlyaem metku 'c' vneshney (opredelena v C++)

.text
GetC:
    ; Korrektno: zagruzit' znachenie po adresu [rip + smeshchenie do 'c']
    mov rax, [c+rip]
    ret
\end{lstlisting}

\subsection{Оптимизация хвостового вызова (TCO)}
Рассмотрим функцию-обертку, которая просто вызывает другую функцию и немедленно возвращает ее результат.
\begin{lstlisting}[language={[x86masm]Assembler}, caption={Неоптимальный хвостовой вызов}, label={lst:bad_tail_call}]
MyFuncWrapper:
    ; ... podgotovka argumentov ...
    call OtherFunc ; 1. Zapisat' adres vozvrata (A) na stek
    ret              ; 2. Snyat' adres (A) so steka i pereyti na nego
\end{lstlisting}
Здесь \texttt{call} кладет на стек адрес возврата (в \texttt{MyFuncWrapper}), а \texttt{ret} немедленно его снимает. Это лишняя работа.

\gls{tco} (\gls{tco}) — это замена \texttt{call} + \texttt{ret} на один \texttt{jmp}.
\begin{lstlisting}[language={[x86masm]Assembler}, caption={Оптимизированный хвостовой вызов (TCO)}, label={lst:good_tail_call}]
MyFuncWrapper:
    ; ... podgotovka argumentov ...
    jmp OtherFunc ; Peredat' upravlenie OtherFunc
\end{lstlisting}
Когда \texttt{OtherFunc} выполнит \texttt{ret}, она вернет управление не в \texttt{MyFuncWrapper}, а тому, кто вызвал \texttt{MyFuncWrapper} (т.к. его адрес возврата все еще лежит на вершине стека). [cite: 608] Компиляторы (\texttt{-O1} и выше) активно применяют эту оптимизацию.

\subsection{Вызов функций C (scanf / printf)}
Пользоваться вводом-выводом C++ (\texttt{iostream}) из \gls{asm} почти невозможно из-за name mangling (искажения имен). [cite: 638-640]
Гораздо проще использовать функции из C \texttt{<stdio.h>}, такие как \texttt{scanf} и \texttt{printf}. [cite: 643]

При этом нужно строго соблюдать два правила соглашения о вызовах (ABI):
\textbf{1. Аргументы:} Первые 6 целочисленных аргументов/указателей передаются через регистры (именно в таком порядке):
\texttt{RDI}, \texttt{RSI}, \texttt{RDX}, \texttt{RCX}, \texttt{R8}, \texttt{R9}.

\textbf{2. Выравнивание стека:} Перед инструкцией \texttt{call} \gls{rsp} (указатель стека) \textbf{должен быть выровнен по 16-байтной границе}. [cite: 602]

\begin{notebox}
\textbf{Ловушка выравнивания:} Когда нашу функцию \texttt{main} вызывают, \gls{rsp} уже выровнен по 16-байтной границе. \textit{Но} инструкция \texttt{call} (которая вызвала \texttt{main}) помещает на стек 8-байтный адрес возврата. [cite: 603]
Это означает, что \textit{внутри} нашей функции \texttt{main} \gls{rsp} \textbf{не выровнен} (он равен $16N + 8$).
Перед тем, как мы сами сделаем \texttt{call} (например, \texttt{call scanf}), мы должны "скомпенсировать" эти 8 байт, например, \texttt{sub rsp, 8}. [cite: 604, 733]
\end{notebox}

\begin{lstlisting}[language={[x86masm]Assembler}, caption={Пример: чтение числа (n) и вывод (n+1) на Assembler}, label={lst:scanf_printf}]
.intel_syntax noprefix

.section .rodata
; Formatnaya stroka dlya chteniya ("%lld")
read_fmt: .asciz "%lld"
; Formatnaya stroka dlya zapisi ("%lld\n")
write_fmt: .asciz "%lld\n"

.text
.global main
main:
    ; --- PROLOG ---
    ; Vydelyaem 8 bayt dlya peremennoy 'n'
    ; I zaoctno VYRAVNIVAEM stek (rsp byl 16N+8, stal 16N)
    sub rsp, 8
    
    ; --- Vyzov scanf ---
    ; scanf("%lld", &n);
    ; &n teper' = adres [rsp]
    
    ; Arg 1 (RDI): Adres formatnoy stroki
    lea rdi, [read_fmt+rip]
    ; Arg 2 (RSI): Adres, kuda pisat' rezul'tat (vershina steka)
    mov rsi, rsp
    
    ; Dlya variadic funktsiy (kak scanf) nuzhno obnulit' rax
    xor rax, rax
    call scanf
    
    ; --- Vyzov printf ---
    ; printf("%lld\n", n + 1);
    ; Zagruzhaem 'n' so steka
    mov rsi, [rsp]
    ; Uvelichivaem
    add rsi, 1
    
    ; Arg 1 (RDI): Adres formatnoy stroki
    lea rdi, [write_fmt+rip]
    ; Arg 2 (RSI): Znachenie (n + 1)
    ; (uzhe v rsi)
    
    xor rax, rax
    call printf

    ; --- EPILOG ---
    ; "return 0;"
    ; Po ABI, my vozvrashchaem znachenie iz main cherez RAX
    xor rax, rax
    
    ; Osvobozhdaem mesto na steke
    add rsp, 8
    ret
\end{lstlisting}

\begin{summarybox}
\begin{itemize}
    \item Для доступа к глобальным данным используйте \textbf{RIP-относительную адресацию} (\texttt{[my\_var+rip]}).
    \item \texttt{call func} + \texttt{ret} можно заменить на \texttt{jmp func} (TCO).
    \item При вызове функций C (e.g., \texttt{printf}) стек \textbf{должен быть выровнен по 16 байт} \textit{до} инструкции \texttt{call}.
    \item Аргументы передаются через \texttt{RDI}, \texttt{RSI}, \texttt{RDX}, \texttt{RCX}...
    \item \texttt{scanf} ожидает \textit{указатель} (адрес) в \texttt{RSI}, \texttt{printf} — \textit{значение}.
    \item Возвращаемое значение из \texttt{main} — это то, что лежит в \texttt{RAX} в момент \texttt{ret}.
\end{itemize}
\end{summarybox}

\section{Синтаксисы ассемблера: Intel vs. AT\&T}
Существует два доминирующих синтаксиса x86 \gls{asm}. 
\begin{itemize}
    \item \textbf{\gls{intel-syntax}:} (Используется в этой лекции, в документации Intel, Microsoft).
    \item \textbf{\gls{att-syntax} (GNU):} (Используется по умолчанию в \texttt{objdump} и \texttt{gcc}). [cite: 688]
\end{itemize}

Ключевые отличия: 
\begin{table}[h]
  \centering
  \caption{Сравнение синтаксисов Intel и AT\&T (GNU)}
  \label{tab:syntax}
  \begin{tabular}{@{}lll@{}}
    \toprule
    Аспект & \textbf{Intel (мы)} & \textbf{AT\&T (GNU)} \\
    \midrule
    Порядок операндов & \texttt{mov rax, rbx} & \texttt{mov \%rbx, \%rax} \\
     & (Приемник, Источник) & (Источник, Приемник) \\
    Регистры & \texttt{rax}, \texttt{rbx} & \texttt{\%rax}, \texttt{\%rbx} (с префиксом \texttt{\%}) \\
    Константы & \texttt{16}, \texttt{0x10} & \texttt{\$16}, \texttt{\$0x10} (с префиксом \texttt{\$}) \\
    Адресация & \texttt{[rax + rbx * 4 + 32]} & \texttt{32(\%rax, \%rbx, 4)} \\
    Размер & \texttt{DWORD PTR [rax]} & \texttt{movl \$0, (\%rax)} (суффикс \texttt{l/q/w/b}) \\
    \bottomrule
  \end{tabular}
\end{table}

\begin{notebox}
Полезно уметь читать оба синтаксиса. В \texttt{objdump} можно включить \gls{intel-syntax} с помощью флага \texttt{-M intel}.
\end{notebox}

% Печать глоссариев (требует: pdflatex -> makeglossaries -> pdflatex -> pdflatex)

% Финальный QC-комментарий (пример для LLM)
% QC: Эстетика — единая цветовая тема, аккуратные заголовки, шапки/футеры, боксы и листинги.
%     Совместимо с pdfLaTeX; minted не используется; TikZ готов; глоссарий печатается.
%     Все окружения закрыты, метки уникальны.
